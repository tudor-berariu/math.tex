\documentclass{beamer}
\usetheme{CambridgeUS}
\beamertemplatenavigationsymbolsempty

\newcommand{\fn}[2][f]{#1\left( #2 \right)}

\newcommand{\fstpd}[2]{\frac{\partial #1}{\partial #2}}
\newcommand{\fstpdfn}[2]{\fn[\frac{\partial}{\partial #2}]{#1}}
\newcommand{\sndpd}[3]{\frac{\partial^2 #1}{\partial #2 \partial #3}}
\newcommand{\sndpdfn}[3]{\fn[\frac{\partial^2}{\partial #2 \partial #3}]{#1}}


\title{math.tex}
\subtitle{a short guide}
\author{Tudor Berariu}
\date{version 0.1}

\begin{document}

\frame[plain]{\titlepage}

\section{Functions}

\begin{frame}[fragile]{Functions}
        \begin{itemize}
            \item Use \verb!\fn! to write \alert{function applications} and not to worry about parantheses.
            \begin{itemize}
                \item \verb!\fn[\max]{x,0}! produces $$\fn[\max]{x, 0}$$
                \item \verb!\fn{\fn[g]{\fn[h]{x^2 + 1}^2}^{-1} - 1}! produces
                  $$\fn[f]{\fn[g]{\fn[h]{x^2 + 1}^2}^{-1} - 1}$$
                \end{itemize}
            \item Define new functions on top of \verb!\fn! for nicer equations.
        \begin{verbatim}
    \newcommand{\foo}[1]{\fn[foo]{#1}}

    \foo{x + 1}
            \end{verbatim}
        \end{itemize}
    \end{frame}
    
    \begin{frame}[fragile]{Partial derivatives}
        \begin{itemize}
            \item Use \verb!\fstpd! and \verb!\fstpdfn! to write \alert{first order partial derivatives}.
                \begin{itemize}
                    \item \verb!\fstpd{f}{x}! produces $$\fstpd{f}{x}$$
                    \item \verb!\fstpdfn{\fn[\sin]{2x + 1}^2 +3}{x}! produces
                     $$\fstpdfn{\fn[\sin]{2x + 1}^2 +3}{x}$$
                \end{itemize}
            \item Similarly, use \verb!\sndpd! and \verb!\sndpdfn! for
             \alert{second order partial derivatives}.
                $$\sndpd{\fn{x, y}}{x}{y} = \sndpdfn{x^2 + 2y}{x}{y}$$
        \end{itemize}        
    \end{frame}

    \begin{frame}[fragile]{Some useful operators}
        \begin{itemize}
            \item Use \verb!\argmax{var}! for the {\tt argmax} operator. \verb!\argmin! exists as well.
                $$\argmax{\lambda} \fn{\lambda}$$
            \item Both \verb!\argmin! and \verb!\argmax! take an optional argument intended to insert the needed space aftet the oprator. The default is \verb!\;!, but you can provide whatever you feel appropriate.
            \begin{itemize}
                \item \verb!\argmin[]{\alpha} \fn{\alpha}! produces
                    $$ \argmin[]{\alpha} \fn{\alpha} $$
                \item \verb!\argmin[\quad]{\alpha} \fn{\alpha}! produces
                    $$ \argmin[\quad]{\alpha} \fn{\alpha} $$
            \end{itemize}
        \end{itemize}
    \end{frame}

    \begin{frame}[fragile]{Expected values}
        \begin{itemize}
            \item Use \verb!\expval! for the usual way of writing expected values.
                \begin{itemize}
                    \item Write \verb!\expval[{x \sim \fn[p]{x}}]{\fn[g]{x}}! to produce
                    \footnote{Yes, this is the correct way to provide optional arguments
                    in \LaTeX: {\tt [\string{...\string}]} }:
                    $$\expval[{x \sim \fn[p]{x}}]{ \fn[g]{x} }$$
                    \item or, simply \verb!\expval{\fn{x}}! to produce:
                    $$\expval{\fn{x}}$$
                \end{itemize}
        \end{itemize}
    \end{frame}

\section{Tensors}
\label{sec:tensors}

    \begin{frame}[fragile]{Matrix operations}
        \begin{itemize}
            \item Use \verb!\tr! to \alert{transpose} matrices (it uses the \verb!\intercal! symbol).
                \begin{itemize}
                    \item \verb!\tr{A}! produces $\tr{A}$.
                \end{itemize}
            \item Use \verb!\inv! to refer to the \alert{inverse} of a matrix.
            \begin{itemize}
                \item \verb!\inv{A}! produces $\inv{A}$.
            \end{itemize}
        \end{itemize}
    \end{frame}

\section{Organizing your formulas}
\label{sec:org}

    \begin{frame}[fragile]{Parentheses and brackets}
        \begin{itemize}
            \item Use \verb!\rp! for \alert{round parentheses} around some expression.
            Do that if you prefer this to writing \verb!\left( ... \right)! yourself.
        \end{itemize}
    \end{frame}

\end{document}
